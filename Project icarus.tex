%% LyX 2.2.1 created this file.  For more info, see http://www.lyx.org/.
%% Do not edit unless you really know what you are doing.
\documentclass[english]{article}
\usepackage[T1]{fontenc}
\usepackage[latin9]{luainputenc}
\usepackage{graphicx}

\makeatletter
%%%%%%%%%%%%%%%%%%%%%%%%%%%%%% Textclass specific LaTeX commands.
\newcommand{\lyxaddress}[1]{
\par {\raggedright #1
\vspace{1.4em}
\noindent\par}
}
\newcommand{\lyxrightaddress}[1]{
\par {\raggedleft \begin{tabular}{l}\ignorespaces
#1
\end{tabular}
\vspace{1.4em}
\par}
}

\makeatother

\usepackage{babel}
\begin{document}

\title{Project Icarus: Crowd Analytics and Management}
\maketitle

\lyxaddress{Ibrahim Mohiuddine, Abbad Vakil, Haytham Shaban, Saransh Kacharia,
Brian Liao}

\lyxrightaddress{Fondry10}
\begin{abstract}
This project explores and applies methods of image analyzation in
several forms in order to monitor the condition and health of a crowd.
Stampedes, congestion, and traffic all occur as a result of inefficient
crowd management. The software developed identifies congested areas
and determines solutions to avoid congestion based on live data. The
data is processed by a local device which is fed via camera. This
method was tested in simulation and proved to create a more efficient
and congestion-free scenario. Future plans include depth sensing for
automatic calibration and suggested course of action.
\end{abstract}

\section{Introduction}

\paragraph{Throughout this paper the crowd analytics software developed by the
Project Icarus team will be described in depth along with its hardware
applications. This paper is targeted at crowd control teams, event
planners, and product developers incorporating the software to various
hardware devices and scenarios.}
\begin{itemize}
\item An overview of Icarus features 
\item A systematic overview of the Icarus software, including processing
methods
\item The analyzation process and methods used to collect the data input
from camera sources
\item The application of live results/suggestions
\end{itemize}

\paragraph{\protect\includegraphics[scale=0.7]{C:/Users/btl78/OneDrive/Pictures/1}}

\subsection{Background}

\paragraph{In the twenty-first century alone, over sixty-one incidents of human
stampedes have occurred across the globe. A commonly overlooked problem,
crowd congestion, is an instantaneous occurrence that can have deadly
consequences, if not managed preemptively. Of these incidents, five
have occurred in Saudi Arabia, primarily during the time of rituals
taking place in Mecca. Mecca\textquoteright s most recent incident
was the 2015 Mina human crush. Due to inefficient crowd management,
over two million pilgrims were escorted to the same location at the
same time of day. A human crush was bound to occur, and over 2,177
pilgrims lost their lives that day.}

\section{What is Project Icarus?}

\paragraph{Icarus is a crowd analytics software designed to pull live data from
a venue/crowd and apply the data in a meaningful way. Project Icarus
utilizes a thermal camera for live frame captures, then processes
the images into data that crowd control teams and event planners can
use to manage crowds more efficiently. Icarus is able to recognize
where crowds of people are located, the size of the crowd, the speed
of the crowd, and tracking of the crowd over time.}

\paragraph{Using the \textquotedblleft FLIR Lepton\textquotedblright{} thermal
camera, frames are captured and analyzed using project Icarus\textquoteright{}
blob detection software. When the program is executed, the \textquotedblleft whiter\textquotedblright{}
parts of the image are detected and ID\textquoteright d with different
colors. This is used to find the size of the crowds and their midpoints,
which are found by averaging the locations of each of the pixels in
a given blob. }

\paragraph{\protect\includegraphics{C:/Users/btl78/OneDrive/Pictures/2}}

\paragraph{Figure 1}

\subsubsection{Density Key}

\paragraph{When a \textquotedblleft blob\textquotedblright{} is recognized,
and the proximity is known, Project Icarus is able to calculate how
much area is occupied per person. The further a person is from the
standard of one square meter, the more saturated its area is in the
color of red. This model provides information for efficient and simplistic
identification.}

\paragraph{For high accuracy, samples are collected one after another and analyzed
repeatedly. The feed received from the camera is processed on the
hardware device and the results are returned to a server. }

\subsection{Feed Information}

\paragraph{Along with receiving data for density from live feed represented
by the coloration, the data collected can be used for several applications.}

\paragraph{\protect\includegraphics[scale=0.8]{C:/Users/btl78/OneDrive/Pictures/3}}

\paragraph{Figure 2}
\begin{itemize}
\item The processed image shows:
\item Congested or highly dense area into sub squares of the larger image
\item The \textbf{green} represents are as with close to no congestion
\item The \textbf{yellow} represents areas with low congestion, potentially
a place where the program will not funnel the crowd
\item The \textbf{red} represents an area which needs to be managed 
\end{itemize}

\subsection{Thermal Camera Identification}

\paragraph{\protect\includegraphics{C:/Users/btl78/OneDrive/Pictures/4}}

\paragraph{Figure 3}

\paragraph{Using Python\textquoteright s OpenCV, Project Icarus is able to take
pictures one after another, and process them consecutively.}

\paragraph{\protect\includegraphics[scale=0.4]{C:/Users/btl78/OneDrive/Pictures/c1.PNG}}

\paragraph{This model allows for tracking moving people along with people at
stand still. Using the thermal feed in conjunction with the analyzation
software allows for thorough results. The thermal feed also monitors
a video feed showing immediate changes in a group of people. This
model is accurate for long range cameras as well as short range ones.}

\paragraph{\protect\includegraphics{C:/Users/btl78/OneDrive/Pictures/5}}

\paragraph{Figure 4}

\paragraph{A recursive algorithm is applied to each pixel that is \textquotedblleft hot\textquotedblright{}
enough to be considered a \textquotedblleft blob\textquotedblright .}

\paragraph{\protect\includegraphics[scale=0.4]{C:/Users/btl78/OneDrive/Pictures/c2.PNG}}

\section{Icarus Feature Set}

\paragraph{Using a \textquotedblleft FLIR Lepton\textquotedblright{} thermal
camera, instances of data retrieved from footage at a capture rate
specified in the program. When the program is executed, the algorithm
scans all heat signatures, incrementing by how much mass a certain
threshold of thermal readings are covering, and calculating the distance
from the crowd to provide an accurate number of people in that area.
The algorithm then individualizes each group and attaches a form of
identification.}

\paragraph{\protect\includegraphics{C:/Users/btl78/OneDrive/Pictures/6}}

\paragraph{Figure 5}

\paragraph*{The main features of Icarus are as follows:}
\begin{itemize}
\item Scans area for heat signatures
\item Groups areas of heat based on how much area of the same threshold
is connected (\textquotedblleft various shapes of different sizes
will appear based on the formation of the crowd)
\item Recognizes each shape as a separate unit
\item Locates congestion points by scanning movement of all shapes and finding
areas moving at a slower pace than expected 
\item Sends information to \textquotedblleft raspberry pi\textquotedblright{}
motherboard, accesses database for layout of intended crowd formation,
and map of the area if stored
\item Compares movement of crowd to expected route, and delivers visual/audial
signals for the crowd to follow accordingly
\end{itemize}

\paragraph*{These features are described in detail in the following sections.}

\section{Project Icarus System Overview}

\paragraph{Project Icarus has four main components aside from the software side.
The thermal camera captures a feed which is processed by the Raspberry
Pi. After processing, the data is sent to a server which packages
the results in a JSon query to be used for the client end.}

\paragraph{\protect\includegraphics[scale=0.8]{C:/Users/btl78/OneDrive/Pictures/7}}

\paragraph{Figure 6}

\paragraph{Thermal Camera Frame Analyzation The thermal camera takes pictures
one after another to track movement of blob objects. Each frame is
parsed through column by column, row by row using Python\textquoteright s
Imaging Library.}

\paragraph{\protect\includegraphics[scale=0.4]{C:/Users/btl78/OneDrive/Pictures/c3.PNG}}

\paragraph{After each iteration of the function, the x value increases moving
across the frame. A flood fill function is called recursively to each
of the neighbors of the scanned pixels which are also in the threshold,
once a pixel with a value out of range is found a new object is created
and stored.}

\subsubsection{Thermal Camera Movement }

\paragraph{Tracking Once each blob\textquoteright s midpoint is calculated,
Project Icarus can track the blob. When the next frame is captured,
the previous midpoint\textquoteright s location is searched, if a
midpoint is found within a certain proximity of the old midpoint\textquoteright s
location, it is ID\textquoteright d the same as the previous midpoint.
Thus allowing Project Icarus to track midpoints, or \textquotedblleft blobs\textquotedblright .}

\subsection{Project Icarus Components}

\subsubsection{\textquotedblleft FLIR Lepton\textquotedblright{} Thermal Camera}

\paragraph{This micro thermal imaging camera works flawlessly in conjunction
with the selected processing hardware.}

\subsubsection{Raspberry Pi}

\paragraph{The Raspberry Pi is a compact, yet powerful computer that handles
all of the image processing sent to it via the thermal camera.}

\subsubsection{Python Server}

\paragraph{A simple Python WebPy server that displays the results of a the program.
The results are fed to an HTML file which decodes the Base64 encoded
image.}

\subsection{Applications}

\paragraph{Several methods for controlling the movement of a group of people
can be applied by the client once all the data is received from our
program. Statistical analysis of crowds, funneling and management
through lights or even people can be integrated seamlessly. }

\paragraph{\protect\includegraphics[scale=0.8]{C:/Users/btl78/OneDrive/Pictures/8}}

\paragraph{Figure 7}

\subsection{User Scenarios}

\subsubsection{Crowd Control Aid}

\paragraph{A peaceful protest ensues as citizens rally against a law recently
passed by the city. As more people begin to gather, the city\textquoteright s
crowd control team begins to monitor their walk across multiple streets.
Icarus is launched, with camera placed at each intersection of interest.
The system frequently collects instances of crowd movement, and after
analyzation, displays the results on the server for all team members
to view. After review of data retrieved, the crowd control team notices
that the people are slowing down turning right on the street through
their route. The team is then able to send members to that location
in order to file them out in a more organized manner, in turn successfully
migrating the crowd, and avoiding collateral damage.}

\subsubsection{Automated Crowd Control System}

\paragraph{A concert is set to take place in a busy part of the city. Hundreds
of people have lined up, and have just check their tickets in at the
entrance. A crowd develops as everyone approaches the door to the
musical hallway. Icarus has already been set up to analyze the approaching
crowd, and begins scanning people, grouping them into grouped sections
based on the area they have taken up. The area is most likely entirely
filled during the beginning, but as crowd begins to file through,
Icarus is consistently searching for congested parts of the crowd.
The system already holds a map of the entrance, and has a calculated
route that would serve as the most efficient way to walk the crowd
through the doors. It signals to individual sides of the crowd to
adjust their speed (or direction if need be), therefore allowing other
sections of the crowd to pass through easily and quickly.}

\subsection{Facilitated VS Un-Facilitated Funneling}

\paragraph{\protect\includegraphics[scale=0.4]{C:/Users/btl78/OneDrive/Pictures/g1}\protect\includegraphics[scale=0.4]{C:/Users/btl78/OneDrive/Pictures/g2}}

\paragraph{Figure 8}

\section{Conclusion}

\paragraph{By analyzing the properties of a crowd and identifying congestion
points, project Icarus can now reform crowd management. By efficiently
scanning the proximity, planning an optimal route, and delivering
signals for people to follow, Stampedes, congestion, and traffic can
now be prevented. With project Icarus capable of functioning as an
individual unit, or as an aid for a crowd control team, this device
will always be a supplement to making crowds safer and faster to travel
in.}

\part*{References}
\begin{enumerate}
\item http://www.huffingtonpost.com/entry/hajj-stampede-death-toll\_us\_5625d226e4b0bce347020aa1
\item http://www.cnn.com/2015/09/25/middleeast/hajj-pilgrimage-stampede/ 
\end{enumerate}

\end{document}
